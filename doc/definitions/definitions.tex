\documentclass[11pt]{article} % use larger type; default would be 10pt

\usepackage[utf8]{inputenc} % set input encoding (not needed with XeLaTeX)

%%% PAGE DIMENSIONS
\usepackage{fullpage}
%\usepackage{geometry} % to change the page dimensions
%\geometry{a4paper} % or letterpaper (US) or a5paper or....
% \geometry{margin=2in} % for example, change the margins to 2 inches all round
% \geometry{landscape} % set up the page for landscape
%   read geometry.pdf for detailed page layout information

\usepackage{graphicx} % support the \includegraphics command and options

% \usepackage[parfill]{parskip} % Activate to begin paragraphs with an empty line rather than an indent

%%% PACKAGES
\usepackage{booktabs} % for much better looking tables
\usepackage{array} % for better arrays (eg matrices) in maths
\usepackage{paralist} % very flexible & customisable lists (eg. enumerate/itemize, etc.)
\usepackage{verbatim} % adds environment for commenting out blocks of text & for better verbatim
\usepackage{subfig} % make it possible to include more than one captioned figure/table in a single float
% These packages are all incorporated in the memoir class to one degree or another...

%%% HEADERS & FOOTERS
\usepackage{fancyhdr} % This should be set AFTER setting up the page geometry
\pagestyle{fancy} % options: empty , plain , fancy
\renewcommand{\headrulewidth}{0pt} % customise the layout...
\lhead{}\chead{}\rhead{}
\lfoot{}\cfoot{\thepage}\rfoot{}

%%% SECTION TITLE APPEARANCE
\usepackage{sectsty}
\allsectionsfont{\sffamily\mdseries\upshape} % (See the fntguide.pdf for font help)
% (This matches ConTeXt defaults)

%%% ToC (table of contents) APPEARANCE
\usepackage[nottoc,notlof,notlot]{tocbibind} % Put the bibliography in the ToC
\usepackage[titles,subfigure]{tocloft} % Alter the style of the Table of Contents
\renewcommand{\cftsecfont}{\rmfamily\mdseries\upshape}
\renewcommand{\cftsecpagefont}{\rmfamily\mdseries\upshape} % No bold!

%%% END Article customizations

%%% The "real" document content comes below...

\title{Brief Article}
\author{\'Eloi Perdereau}

\begin{document}
\maketitle

\section{Notations}

\paragraph{Chemin}
Soit $p$ et $q$ deux points appartenant au complexe. On note $\lambda (p, q)$ la liste ordonnée des arêtes constituant le plus court chemin reliant $p$ à $q$.

\section{Définitions}

On considerera le point $o$ qui est le point du complexe dominant tout les autres.

\paragraph{Domination}
Soit $a$ et $b$ deux points de coordonnées respectives $(x_a, y_a, z_a)$ et $(x_b, y_b, z_b)$. On dit que $a$ domine $b$ si $x_a \leq x_b$, $y_a \leq y_b$ et $z_a \leq z_b$.

\paragraph{Prolongement interne}
Soit $p$ et $q$ deux points appartenant au complexe. On dit que $q$ est dans le prolongement interne de $\lambda (o, p)$ si $\lambda (o, p) \subset \lambda (o, q)$.

\paragraph{Prolongement direct}
Soit $p$ et $q$ deux points tels que $q$ est dans le prolongement interne de $\lambda (o, p)$. On dit que $q$ est dans le prolongement direct de $\lambda (o, p)$ si $\lambda (p, q)$ n'est constitué que d'un seul segment.

\paragraph{Prolongement externe}
Soit $p$ un point à l'intérieur du complexe et $q$ un point à l'extérieur. $q$ est dans le prolongement externe de $\lambda (o, p)$ s'il existe un point $q' \in [pq]$ tel que $q'$ est dans le prolongement direct de $\lambda (o, p)$.

\paragraph{Prolongement vers une droite}
Soit $p$ un point du complexe, et $d$ une droite axe-parallèle (à définir) qui intersecte des points dominés par $p$. Le prolongement de $\lambda (o, p)$ vers $d$ est un point $q \in d$ tel que $q$ est dans le prolongement interne ou externe de $\lambda (o, p)$. Si $q$ existe, il est unique.

\paragraph{Remarque} Il peut y avoir plusieurs prolongement d'un chemin $\lambda (o, p)$ vers $d$.

\paragraph{Extension}
Soit $\lambda_1$ et $\lambda_2$ deux chemins dans le complexe. On dit que $\lambda_2$ est une extension de $\lambda_1$ si $\lambda_1 \subset \lambda_2$.

\paragraph{Extension unitaire}
Soit $\lambda_1$ et $\lambda_2$ deux chemins dans le complexe tels que $\lambda_2$ est une extension de $\lambda_1$. On dit que $\lambda_2$ est une extension unitaire de $\lambda_1$ si $\lambda_2 - \lambda_1$ n'est constitué que d'un seul segment.

\section{Calcul du prolongement d'un chemin vers une droite}

\'Etant donné $p$ un point appartenant à une et une seule arête $e$ concave du complexe, et soit une droite $d$ axe-parallèle qui intersecte des points dominés par $p$. On veut calculer le point $q$ qui est le prolongement de $\lambda (o, p)$ vers $d$. Soit $[rp]$ le dernier segment de $\lambda (o, p)$. Par définition $r$ domine $p$. \\

On considère le plan $P$ contenant $[rp]$ et $e$ et on note $q_0$ le point d'intersection de $P$ et $d$. \\
On considère le plan $Q$ à plat par rapport à l'obstacle de $e$ et on note $q_1$ le point d'intersection de $Q$ et $d$. \\
$q \in [q_0 q_1]$

\end{document}






